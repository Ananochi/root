-----------------------------------------------------------------------------------------
********************************** Emergency - IoT **************************************
-----------------------------------------------------------------------------------------

One of the challenges with the IoT architecture, is in case of Power and/or network failure.

Imagine the system at the given smart home is not able to communicate with the "cloud",
one could be afraid that the system might not be able to report e.g. a fire detection.

Obviously the given house have "classic/analog" firedetectors, but as long as the Process
of calling the correct Fire Department is located in the "cloud", this could create some
challenges.

Concerning solution to this problem, we see the following three scenarios of implementing
this IoT smart home "failsafe". These are the following:

- Inhouse emergency routine storing

	The given emergency processes is stored locally in the smart home (in some device). 
	This would mean that even though no network access was available, the system would 
	be able to take the needed actions
	[TODO] This would create some difficulties with support and maintenance (even security!!!!: safety gets old ..)
	
- Hybrid emergency routine storing

	The emergency processes are stored locally AND in the "cloud". BLABLABLA
	
- Cloud emergency routine storing

	Everything is in the cloud. You have UPS's to manage power failures! Done.



